Topological superconductivity is a relatively fresh topic of condensed matter physics. Being a rich platform for intriguing and beautiful problems, it also has a huge and unrevealed potential for technology, especially for quantum computing.

The notion of topological superconductivity is closely related to a possibility of presence of a Majorana state --- special topologically protected state, usually localized near some topological defect in a topological superconductor. Despite  numerous theoretical proposals of constructing this state in condensed matter, its observation in experimental setup is still a big challenge

In this work the system of two one-dimensional superconducting wires connected with a tunnel junction is considered. Under special conditions this system can host a Majorana fermion. The properties of this system, such as subgap states, stationary supercurrent and ionization of the Majorana state under  oscillating external voltage are studied. The results of this work have the potential in developing a new technique of detecting Majorana fermions in such systems.