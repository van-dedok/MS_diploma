Топологическая сверхпроводимость является относительно новой областью современной физики. Являясь богатым полем для красивых теоретических задач, она также имеет огромный и нереализованный потенциал для использования в практических целях, в особенности для квантовых вычислений


Понятие топологической сверхпроводимости тесно связано с существованием Майорановского состояния --- особенного топологически защищенного квантового состояния, обычно локализованного вблизи некоторой неоднородности в топологическом сверхпроводнике. Несмотря на многочисленные теоретические предложения по созданию этого состояния в твердых телах, его экспериментальное наблюдение все ещё остается сложной и неразработанной задачей

В этой работе рассмотрена система, состоящая из двух одномерных сверхпроводников, соединенных туннельным контактом. При определенных условиях такая сисемта может иметь Майрановскоое состояние, локализованное вблизи барьера. В работе изучено наличие подщелевых состояний стационарный сверхток и ионизация под действием внешнего непрерывного излучения. Ответы, полученные в ходе выяснения данных вопросов, имеют потенциальное применение для детектирования Майорановских состояний в подобных системах.