\chapter{Multiphoton ionization} 
\label{app:multiphoton ionization}
This appendix is focused on high-order perturbation theory, ionization rates in particular. 

\section{Basics about Green's functions}

The starting point is the general setup of a discrete bound state subject to a weak and slow perturbation $ V(t) $.The bound state energy is zero. The goal is to obtain the ionization rate. The time evolution of the wave function obeys the Schroedinger equation:
\begin{gather}
\label{shr_eq_general}
	i\frac{\partial}{\partial t}\Psi=(H_{0}+V)\Psi
\end{gather}

In the absence of perturbations, the solution is $ \Psi(t)=\Psi_{0} $ (since $ E_{0}=0 $ it is literally time-independent). For further analysis it's convenient to consider the unperturbed retarded Green's function $ G^{R}(E) $ defined so that:
\begin{gather}
\label{green_function_def}
	G^{R}(E)(E+i0-H_{0})=1
\end{gather}
If the bound state is normalized, $ \langle\gamma|\gamma\rangle= $1 and the continuous spectrum is normalized according to $ \langle E|E'\rangle=N(E)\delta(E-E') $ with some reasonably nice $ N(E $) then:
\begin{gather}
\mathbb{I}=|\gamma\rangle\langle\gamma|+\int\frac{|E\rangle\langle E|}{N(E)}dE
\end{gather}
Similarly, $ H_{0} $ and $ G^{R} $ in the energy representation:
\begin{gather}
	H_{0}=\int\frac{|E\rangle\langle E|}{N(E)}EdE
	\qquad
	G^{R}(\epsilon)=\frac{|\gamma\rangle\langle\gamma|}{\epsilon+i0}+\int\frac{|E\rangle\langle E|}{(\epsilon+i0-E)N(E)}dE
\end{gather}
Integrals over $ E $ include all states in the continuous spectrum. Where the latter is degenerate, a summation over the degenerate states must be carried out. From now on the "$ R $" index for retarded Green's function will be omitted.

In time representation the Green's function obeys:
\begin{gather}
	\left(i\frac{\partial}{\partial t_{2}}-H_{0}\right)G(t_{2},t_{1})=\delta(t_{2}-t_{1})
\end{gather}
One can formally introduce the Green's function in energy representation with two arguments as a Fourier-transform of $ G(t_{2},t_{1}) $:
\begin{gather}
	G\br{E_2,E_1}
	=
	\iint
	e^{iE_2t_2-iE_1t_1}
	G(t_{2},t_{1})
	dt_1
	dt_2
\end{gather}
As $ H_0 $ is time independent, $ G\br{t_2,t_1}=G\br{t_2-t_1,0} $, so:
\begin{gather}
	G\br{E_2,E_1}=2\pi \delta\br{E_2-E_1} G\br{E_1}
\end{gather}
One can also find, that for any eigenstate $ \ket{E_0} $
\begin{gather}
	G\br{t}\ket{E_0}
	=
	-ie^{-iE_0t}\theta_H\br{t}\ket{E_0}
\end{gather}
With the help of Green's function the Schroedinger equation (\ref{shr_eq_general}) can be solved as:
\begin{multline}
\label{general_expression_for the_wavefunction}
\Psi(t)=\Psi_{0}+\int G_{0}(t-t')V(t')\Psi_{0}(t')dt'+
\\
\iint G_{0}(t-t')V(t')G_{0}(t'-t'')V(t'')\Psi_{0}(t'')dt'dt''+\dots
\end{multline}
where the $ \Psi_0 $ is unperturbed wavefunction. This equation can be rewriteen  by the introducion 
\section{Fermi Golden Rule (first order)}

Consider first the lowest order to the Fermi Golden rule by only keeping the linear term in $ V $. Let the Fourier decomposition of $ V $ be:
\begin{gather}
\label{fourier_pertrubation}
	V(t)=\sum_n V_{n}e^{-i\omega_{n}t}
\end{gather}
(Hermiticity demands $ V(t)=V(t)^{*} $ so that $ V_{n}=V_{-n}^{*} $). Suppose there is an unperturbed continuous spectrum parameterized by $ E $, with $ \langle E|E'\rangle=f(E)\delta(E-E') $. The first step is to calculate $ \langle E|\Psi(t)\rangle $, i.e. the quantum amplitude of being in state $ |E\rangle $ at time $ t $. It is:
\begin{multline}
	\langle E|\Psi(t)\rangle=\langle E|\int G_{0}(t-t')V(t')\Psi_{0}(t')dt'\rangle=\int\langle E|G_{0}(t-t')V(t')|\Psi_{0}\rangle dt'=\\=\int\int\langle E|G_{0}(t-t')|E'\rangle\langle E'|V(t')|\Psi_{0}\rangle dt'\frac{dE'}{N(E)}
	\\=\int e^{-i(t-t')E}\theta_{H}(t-t')\langle E|V(t')|\Psi_{0}\rangle dt'=\\=e^{-iEt}\sum_{n}\int e^{it'(E-\omega_{n})}\theta_{H}(t-t')\langle E|V_{n}|\Psi_{0}\rangle dt'
\end{multline}

Assuming that the perturbation was turned on at $ t'=0 $ the integral over $ t' $ is taken from $ t_{0} $ to $ t $ and gives:
\begin{gather}
	\langle E|\Psi(t)\rangle=e^{-iEt}\sum_{n}\langle E|V_{n}|\Psi_{0}\rangle\frac{e^{it(E-\omega)}-1}{i(E-\omega_{n})}
\end{gather}

Thus, the probability density of being in the state $ |E\rangle $ is (omit all frequencies except the resonant on should be ommited since the contributions from non-resonant frequencies can be neglected at long times):
\begin{gather}
	\rho(E)=|\langle E|V_{n}|\Psi_{0}\rangle|^{2}\frac{4\sin^{2}\frac{t(E-\omega_{n})}{2}}{(E-\omega_{n})^{2}}
\end{gather}
so that the probability of being in a state between $ E $ and $ E+\delta E $ is at large times:

\begin{multline}
	P(E+\delta E,E)=|\langle E|V_{n}|\Psi_{0}\rangle|^{2}\intop_{E}^{E+\delta E}\frac{dE}{N(E)}\frac{4\sin^{2}\frac{t(E-\omega_{n})}{2}}{(E-\omega_{n})^{2}}=\\=2t|\langle E|V_{n}|\Psi_{0}\rangle|^{2}\intop_{E}^{E+\delta E}\frac{d(tE/2)}{N(E)}\frac{4\sin^{2}\frac{t(E-\omega_{n})}{2}}{t^{2}(E-\omega_{n})^{2}}=\\=2t|\langle E|V_{n}|\Psi_{0}\rangle|^{2}\frac{\pi}{N(E)}\theta_{H}(\omega_{n}-E)\theta_{H}(E+\delta E-\omega_{n})
\end{multline}
In other words, the probability density at large t behaves exactly like the $ \delta $-function:
\begin{gather}
	\rho(E)=2\pi|\langle E|V_{n}|\Psi_{0}\rangle|^{2}\delta(E-\omega_{n})
\end{gather} 
which is the well-known Fermi Golden rule. (Note, that the probability is $ dP=\rho(E)dE/f(E) $ – do not forget the normalization). Thus, the ionization rate (i.e. $ P/t $) for a single photon is expressed as:
\begin{gather}
\label{golden_reul_first_order}
	\mathcal{I}=\frac{2\pi|\langle E|V_{n}|\Psi_{0}\rangle|^{2}}{f(E)}
\end{gather}
\section{Fermi Golden Rule (high order)}
Tho treat the higher-order perturbation theory it's convenient to rewrite (\ref{general_expression_for the_wavefunction}) as:
\begin{gather}
	\Psi(t)=\Psi_{0}+\int\int G_{0}(t-t')W(t',t'')\Psi_{0}(t'')dt'dt''
\end{gather}
where W incorporates all powers of perturbation theory:
\begin{gather}
	W=V+VG_0V+VG_0VG_0V+\dots
\end{gather}
The perturbation $ V $ in energy space is:
\begin{multline}
	V(E',E)\equiv\int V(t',t)e^{it'E'-iEt}dt'dt=\int V(t)e^{i(E'-E)t}dt=\\=\sum_{n}V_{n}\int e^{i(E'-E-\omega_{n})t}dt=\sum_{n}V_{n}2\pi\delta(E'-E-\omega_{n})
\end{multline}
The second term of the perturbation theory $ W_{2}=VGV $ in energy representation reads:
\begin{multline}
	W_{2}(E_{2},E_{1})=
	\int V(E_{2},E)G_0(E,E')V(E',E_{1})\frac{dEdE'}{(2\pi)^{2}}=
	\\
	\sum_{nm}2\pi\delta(E_{2}-E_{1}-\omega_{n}-\omega_{m})V_{n}G_{0}(E_{1}+\omega_{m})V_{m}
\end{multline} 
Very similarly, the $ N $-th-order term is:
\begin{gather}
	W_{N}(E_{2},E_{1})=\sum_{n_{1},\dots,n_{N}}2\pi\delta\left(E_{2}-E_{1}-\sum_{i=1}^{N}\omega_{n_{i}}\right)V_{n_{N}}\prod_{j=1}^{N-1}G_{0}\left(E_{1}+\sum_{s=1}^{j}\omega_{n_{s}}\right)V_{n_{j}}
\end{gather}

The above expression sums over all processes involving N photons. Note how the Green's functions contains the cumulative energy of all photons absorbed by the time. The $ \delta $-function at the start of the expression indicates that the energy is changed by $ \sum\omega_{n_{i}} $. It makes more sense to sort processes within $ W $ not by photon count $ N $ but rather by the total energy gained, since the total ionization rate should be dominated by ionization to the lowest accessible continuum state. Defining this activation energy as $ \mathcal{E} $, write:
\begin{gather}
W_{\mathcal{E}}(E_{2},E_{1})=2\pi\delta(E_{2}-E_{1}-\mathcal{E})w_{\mathcal{E}}(E_{1})\\
\label{ionization_key}
w_{\mathcal{E}}(E)
=
\sum_{\{n_{i}\}:\mathcal{E}}V_{n_{N}}\prod_{j=1}^{N-1}G_{0}\left(E+\sum_{s=1}^{j}\omega_{n_{s}}\right)V_{n_{j}}
\end{gather}
The summation is over all sets of photons $ {n_{i}} $ that total an energy of $ \mathcal{E} $, i.e. such sets that $ \sum_{i}^{N}\omega_{n_{i}}=\mathcal{E} $. The photon number N itself depends on the particular photon set $ n_{i} $. The total composite perturbation $ W $ can be written as a sum of $ W_{\mathcal{E}} $ with different $ \mathcal{E} $. However, only one term is relevant --- the one with $ \mathcal{E}=\min{|g_{L}|,|g_{L}| }$. (If the elementary frequency quantum $ \omega $ is large enough, this is replaced by the lowest $ \mathcal{E} $ that surpasses the minimum gap). 

Now  the Fermi Golden rule result (\ref{golden_reul_first_order}) can be rederived 	for $ W_{\mathcal{E}} $. It's easy ti see that $ W $ has the same energy structure as the single-photon perturbation $ V $. Both operators simply shift the energy. Thus, one may simply put $ W_{\mathcal{E}} $ into the Fermi Golden rule instead of $ V $ to get the full-theory results:
\begin{gather}
	\mathcal{I}=\frac{2\pi|\langle\mathcal{E}|w_{\mathcal{E}}|\Psi_{0}\rangle|^{2}}{N(\mathcal{E})}
\end{gather}
Thus to find the ionization rate one should calculate $ w_{\mathcal{E}} $ from (\ref{ionization_key}).