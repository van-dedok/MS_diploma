%% This is an example first chapter.  You should put chapter/appendix that you
%% write into a separate file, and add a line \include{yourfilename} to
%% main.tex, where `yourfilename.tex' is the name of the chapter/appendix file.
%% You can process specific files by typing their names in at the 
%% \files=
%% prompt when you run the file main.tex through LaTeX.
\chapter{Introduction}


The system, considered in this work, is a pair of 1D superconductors connected with a Josephson junction. For all the discussion presented it's crucial for one of superconductors to be topological. 

Topological superconductivity is relatively fresh topic in physics. On the one hand it's being connected to particle physics through the notion of Majorana fermion -- the particle coinciding with it's own antiparticle\cite{Majorana_1937}. It can be looked for not only in Standart models' context\cite{particle_majorana_Avignone}\cite{particle_majorana_Giuliani}\cite{particle_majorana_Marcocci}, but also as a state in solids\cite{majorana_condmat_Rossi}\cite{majorana_condmat_Kitaev}\cite{majorana_condmat_Kopnin}\cite{majorana_condmat_Motrunich}\cite{majorana_condmat_Nayak}\cite{majorana_condmat_Read_Green}\cite{majorana_condmat_Senthil}\cite{majorana_condmat_Volovik}\cite{majorana_condmat_Fu_Kane}\cite{review_majorana_Aguado}\cite{review_majorana_Beenakker}\cite{review_majorana_Oppen}. Despite the difference between these entities, there is a clear analogy between majoranas in condensed matter and majoranas in particle physics\cite{Dirak_BdG_Chamon}\cite{Dirak_BdG_Elliott}.

 On the other hand topological superconductivity is of interest to quantum computation community as a platform to build fault tolerant quantum memory\cite{majorana_condmat_Kitaev}\cite{quintum_computation_Alicea}\cite{quintum_computation_Nayak}\cite{quintum_computation_Romito}. Although significant difficulties has appeared on this way, the intention to realize this program is still strong and gives the motivation to build a superconducting samples, which demonstrates signatures of nontrivial topology and presence of Majorana fermions\cite{majorana_experiment_Kouwenhoven}\cite{majorana_experiment_Vaitiekėnas}\cite{majorana_experiment_Zhang}.
 
The proposition of using superconducting wires as a carriers of Majorana fermions came from a seminal work of Kitaev \cite{majorana_condmat_Kitaev}. The key ingredient of this system was a p-wave superconductivity assumed to be present in a wire. There was showed, that under certain conditions the Majroana state ca be present at the end of the wire. After some time another propositions\cite{Oreg_2010}\cite{Lutchyn_2010} appeared, based on seminconductor-superconductor heterostructures with s-wave superconductivity, external magnetic field and spin-orbit coupling. It was showed, that the sign of quantity $ g=B=\sqrt{B^2+\mu^2} $, where $ B $ is a magnetic field, $ \Delta $ is the absolute value of superconducting order parameter and $ \mu $ is a chemical potential can be used as a topological index, and a Majorana state will appear where $ \mathrm{sgn}~g $ is changing.

The model, considered is this work is close to the ones used in \cite{Oreg_2010}\cite{Lutchyn_2010}. It consists of two superconducting wires connected with a tunnel junction. However instead of domain wall of $ \mathrm{sgn}~g $, a tunnel barrier between areas with $ g>0 $ and $ g<0 $ is introduced. This model formulated in detail in chapter \ref{chap:model}. The spectrum of this model and stationary supercurrent are studied in chapter \ref{chap:stationary} and the ionization under small external oscillating voltage is considered in chapter \ref{chap:ionization}. Chapter \ref{chap:discssion}  stands for the discussion of obtained results and their possible experimental realization and use, while chapter \ref{chap:conclision} to conclusion chapter concludes the study.


