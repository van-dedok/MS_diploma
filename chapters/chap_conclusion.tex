\chapter{Conclusion}
\label{chap:conclision}
In this work we have considered the system of two superconducting wires connected with a tunnel junction. A strong spin-orbit and a magnetic field perpendicular to the wire were assumed. The transparency of the barrier were set to be weak, so the system operates in tunneling regime. The model is introduced in detail in chapter \ref{chap:model}.

The low-energy spectrum was obtained for different topological indexes of the wires. The subgap states, found in section \ref{sec:Subgap_states}, are quite predictable.  In triv-top contact there is only one subgap state --- a Majorana state, on zero energy, as it should be. In top-top contact there are two subgap states, which are Majorana states, each from its own wire. As there is a finite transparency of the barrier, these states are not at zero energy and demonstrate the energy splitting, calculated in section \ref{subsect: weak_tunneling}. In triv-triv contact there are no subgape states --- this result, as well as the presence of only Majorana states in other contacts, wasn't obvious for us before, but also not especially surprising.

The supercurrent from low energy states was calculated in section \ref{sec:stationary_supercurrent}. The main result is that it is expected be much smaller than the current from high energy states and probably won't be observable.

In chapter \ref{chap:ionization} the model was modified by introducing time dependent perturbation. Even the simple case, when gap in the trivial wire is much smaller than the gap in topological one and the ionization amplitude factorizes, demonstrates a rich physics with four different subregimes. The ionization rates for Majorana state for these subregimes were calculated, as well as the limits of applicability and their physical meaning is established.

Authors hope, that this work can give further incites for both developing experimental techniques of detecting Majorana states in 1D superconductors and theoretical studies of properties of such systems.